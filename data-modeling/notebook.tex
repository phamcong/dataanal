
% Default to the notebook output style

    


% Inherit from the specified cell style.




    
\documentclass[11pt]{article}

    
    
    \usepackage[T1]{fontenc}
    % Nicer default font (+ math font) than Computer Modern for most use cases
    \usepackage{mathpazo}

    % Basic figure setup, for now with no caption control since it's done
    % automatically by Pandoc (which extracts ![](path) syntax from Markdown).
    \usepackage{graphicx}
    % We will generate all images so they have a width \maxwidth. This means
    % that they will get their normal width if they fit onto the page, but
    % are scaled down if they would overflow the margins.
    \makeatletter
    \def\maxwidth{\ifdim\Gin@nat@width>\linewidth\linewidth
    \else\Gin@nat@width\fi}
    \makeatother
    \let\Oldincludegraphics\includegraphics
    % Set max figure width to be 80% of text width, for now hardcoded.
    \renewcommand{\includegraphics}[1]{\Oldincludegraphics[width=.8\maxwidth]{#1}}
    % Ensure that by default, figures have no caption (until we provide a
    % proper Figure object with a Caption API and a way to capture that
    % in the conversion process - todo).
    \usepackage{caption}
    \DeclareCaptionLabelFormat{nolabel}{}
    \captionsetup{labelformat=nolabel}

    \usepackage{adjustbox} % Used to constrain images to a maximum size 
    \usepackage{xcolor} % Allow colors to be defined
    \usepackage{enumerate} % Needed for markdown enumerations to work
    \usepackage{geometry} % Used to adjust the document margins
    \usepackage{amsmath} % Equations
    \usepackage{amssymb} % Equations
    \usepackage{textcomp} % defines textquotesingle
    % Hack from http://tex.stackexchange.com/a/47451/13684:
    \AtBeginDocument{%
        \def\PYZsq{\textquotesingle}% Upright quotes in Pygmentized code
    }
    \usepackage{upquote} % Upright quotes for verbatim code
    \usepackage{eurosym} % defines \euro
    \usepackage[mathletters]{ucs} % Extended unicode (utf-8) support
    \usepackage[utf8x]{inputenc} % Allow utf-8 characters in the tex document
    \usepackage{fancyvrb} % verbatim replacement that allows latex
    \usepackage{grffile} % extends the file name processing of package graphics 
                         % to support a larger range 
    % The hyperref package gives us a pdf with properly built
    % internal navigation ('pdf bookmarks' for the table of contents,
    % internal cross-reference links, web links for URLs, etc.)
    \usepackage{hyperref}
    \usepackage{longtable} % longtable support required by pandoc >1.10
    \usepackage{booktabs}  % table support for pandoc > 1.12.2
    \usepackage[inline]{enumitem} % IRkernel/repr support (it uses the enumerate* environment)
    \usepackage[normalem]{ulem} % ulem is needed to support strikethroughs (\sout)
                                % normalem makes italics be italics, not underlines
    

    
    
    % Colors for the hyperref package
    \definecolor{urlcolor}{rgb}{0,.145,.698}
    \definecolor{linkcolor}{rgb}{.71,0.21,0.01}
    \definecolor{citecolor}{rgb}{.12,.54,.11}

    % ANSI colors
    \definecolor{ansi-black}{HTML}{3E424D}
    \definecolor{ansi-black-intense}{HTML}{282C36}
    \definecolor{ansi-red}{HTML}{E75C58}
    \definecolor{ansi-red-intense}{HTML}{B22B31}
    \definecolor{ansi-green}{HTML}{00A250}
    \definecolor{ansi-green-intense}{HTML}{007427}
    \definecolor{ansi-yellow}{HTML}{DDB62B}
    \definecolor{ansi-yellow-intense}{HTML}{B27D12}
    \definecolor{ansi-blue}{HTML}{208FFB}
    \definecolor{ansi-blue-intense}{HTML}{0065CA}
    \definecolor{ansi-magenta}{HTML}{D160C4}
    \definecolor{ansi-magenta-intense}{HTML}{A03196}
    \definecolor{ansi-cyan}{HTML}{60C6C8}
    \definecolor{ansi-cyan-intense}{HTML}{258F8F}
    \definecolor{ansi-white}{HTML}{C5C1B4}
    \definecolor{ansi-white-intense}{HTML}{A1A6B2}

    % commands and environments needed by pandoc snippets
    % extracted from the output of `pandoc -s`
    \providecommand{\tightlist}{%
      \setlength{\itemsep}{0pt}\setlength{\parskip}{0pt}}
    \DefineVerbatimEnvironment{Highlighting}{Verbatim}{commandchars=\\\{\}}
    % Add ',fontsize=\small' for more characters per line
    \newenvironment{Shaded}{}{}
    \newcommand{\KeywordTok}[1]{\textcolor[rgb]{0.00,0.44,0.13}{\textbf{{#1}}}}
    \newcommand{\DataTypeTok}[1]{\textcolor[rgb]{0.56,0.13,0.00}{{#1}}}
    \newcommand{\DecValTok}[1]{\textcolor[rgb]{0.25,0.63,0.44}{{#1}}}
    \newcommand{\BaseNTok}[1]{\textcolor[rgb]{0.25,0.63,0.44}{{#1}}}
    \newcommand{\FloatTok}[1]{\textcolor[rgb]{0.25,0.63,0.44}{{#1}}}
    \newcommand{\CharTok}[1]{\textcolor[rgb]{0.25,0.44,0.63}{{#1}}}
    \newcommand{\StringTok}[1]{\textcolor[rgb]{0.25,0.44,0.63}{{#1}}}
    \newcommand{\CommentTok}[1]{\textcolor[rgb]{0.38,0.63,0.69}{\textit{{#1}}}}
    \newcommand{\OtherTok}[1]{\textcolor[rgb]{0.00,0.44,0.13}{{#1}}}
    \newcommand{\AlertTok}[1]{\textcolor[rgb]{1.00,0.00,0.00}{\textbf{{#1}}}}
    \newcommand{\FunctionTok}[1]{\textcolor[rgb]{0.02,0.16,0.49}{{#1}}}
    \newcommand{\RegionMarkerTok}[1]{{#1}}
    \newcommand{\ErrorTok}[1]{\textcolor[rgb]{1.00,0.00,0.00}{\textbf{{#1}}}}
    \newcommand{\NormalTok}[1]{{#1}}
    
    % Additional commands for more recent versions of Pandoc
    \newcommand{\ConstantTok}[1]{\textcolor[rgb]{0.53,0.00,0.00}{{#1}}}
    \newcommand{\SpecialCharTok}[1]{\textcolor[rgb]{0.25,0.44,0.63}{{#1}}}
    \newcommand{\VerbatimStringTok}[1]{\textcolor[rgb]{0.25,0.44,0.63}{{#1}}}
    \newcommand{\SpecialStringTok}[1]{\textcolor[rgb]{0.73,0.40,0.53}{{#1}}}
    \newcommand{\ImportTok}[1]{{#1}}
    \newcommand{\DocumentationTok}[1]{\textcolor[rgb]{0.73,0.13,0.13}{\textit{{#1}}}}
    \newcommand{\AnnotationTok}[1]{\textcolor[rgb]{0.38,0.63,0.69}{\textbf{\textit{{#1}}}}}
    \newcommand{\CommentVarTok}[1]{\textcolor[rgb]{0.38,0.63,0.69}{\textbf{\textit{{#1}}}}}
    \newcommand{\VariableTok}[1]{\textcolor[rgb]{0.10,0.09,0.49}{{#1}}}
    \newcommand{\ControlFlowTok}[1]{\textcolor[rgb]{0.00,0.44,0.13}{\textbf{{#1}}}}
    \newcommand{\OperatorTok}[1]{\textcolor[rgb]{0.40,0.40,0.40}{{#1}}}
    \newcommand{\BuiltInTok}[1]{{#1}}
    \newcommand{\ExtensionTok}[1]{{#1}}
    \newcommand{\PreprocessorTok}[1]{\textcolor[rgb]{0.74,0.48,0.00}{{#1}}}
    \newcommand{\AttributeTok}[1]{\textcolor[rgb]{0.49,0.56,0.16}{{#1}}}
    \newcommand{\InformationTok}[1]{\textcolor[rgb]{0.38,0.63,0.69}{\textbf{\textit{{#1}}}}}
    \newcommand{\WarningTok}[1]{\textcolor[rgb]{0.38,0.63,0.69}{\textbf{\textit{{#1}}}}}
    
    
    % Define a nice break command that doesn't care if a line doesn't already
    % exist.
    \def\br{\hspace*{\fill} \\* }
    % Math Jax compatability definitions
    \def\gt{>}
    \def\lt{<}
    % Document parameters
    \title{assignment4}
    
    
    

    % Pygments definitions
    
\makeatletter
\def\PY@reset{\let\PY@it=\relax \let\PY@bf=\relax%
    \let\PY@ul=\relax \let\PY@tc=\relax%
    \let\PY@bc=\relax \let\PY@ff=\relax}
\def\PY@tok#1{\csname PY@tok@#1\endcsname}
\def\PY@toks#1+{\ifx\relax#1\empty\else%
    \PY@tok{#1}\expandafter\PY@toks\fi}
\def\PY@do#1{\PY@bc{\PY@tc{\PY@ul{%
    \PY@it{\PY@bf{\PY@ff{#1}}}}}}}
\def\PY#1#2{\PY@reset\PY@toks#1+\relax+\PY@do{#2}}

\expandafter\def\csname PY@tok@w\endcsname{\def\PY@tc##1{\textcolor[rgb]{0.73,0.73,0.73}{##1}}}
\expandafter\def\csname PY@tok@c\endcsname{\let\PY@it=\textit\def\PY@tc##1{\textcolor[rgb]{0.25,0.50,0.50}{##1}}}
\expandafter\def\csname PY@tok@cp\endcsname{\def\PY@tc##1{\textcolor[rgb]{0.74,0.48,0.00}{##1}}}
\expandafter\def\csname PY@tok@k\endcsname{\let\PY@bf=\textbf\def\PY@tc##1{\textcolor[rgb]{0.00,0.50,0.00}{##1}}}
\expandafter\def\csname PY@tok@kp\endcsname{\def\PY@tc##1{\textcolor[rgb]{0.00,0.50,0.00}{##1}}}
\expandafter\def\csname PY@tok@kt\endcsname{\def\PY@tc##1{\textcolor[rgb]{0.69,0.00,0.25}{##1}}}
\expandafter\def\csname PY@tok@o\endcsname{\def\PY@tc##1{\textcolor[rgb]{0.40,0.40,0.40}{##1}}}
\expandafter\def\csname PY@tok@ow\endcsname{\let\PY@bf=\textbf\def\PY@tc##1{\textcolor[rgb]{0.67,0.13,1.00}{##1}}}
\expandafter\def\csname PY@tok@nb\endcsname{\def\PY@tc##1{\textcolor[rgb]{0.00,0.50,0.00}{##1}}}
\expandafter\def\csname PY@tok@nf\endcsname{\def\PY@tc##1{\textcolor[rgb]{0.00,0.00,1.00}{##1}}}
\expandafter\def\csname PY@tok@nc\endcsname{\let\PY@bf=\textbf\def\PY@tc##1{\textcolor[rgb]{0.00,0.00,1.00}{##1}}}
\expandafter\def\csname PY@tok@nn\endcsname{\let\PY@bf=\textbf\def\PY@tc##1{\textcolor[rgb]{0.00,0.00,1.00}{##1}}}
\expandafter\def\csname PY@tok@ne\endcsname{\let\PY@bf=\textbf\def\PY@tc##1{\textcolor[rgb]{0.82,0.25,0.23}{##1}}}
\expandafter\def\csname PY@tok@nv\endcsname{\def\PY@tc##1{\textcolor[rgb]{0.10,0.09,0.49}{##1}}}
\expandafter\def\csname PY@tok@no\endcsname{\def\PY@tc##1{\textcolor[rgb]{0.53,0.00,0.00}{##1}}}
\expandafter\def\csname PY@tok@nl\endcsname{\def\PY@tc##1{\textcolor[rgb]{0.63,0.63,0.00}{##1}}}
\expandafter\def\csname PY@tok@ni\endcsname{\let\PY@bf=\textbf\def\PY@tc##1{\textcolor[rgb]{0.60,0.60,0.60}{##1}}}
\expandafter\def\csname PY@tok@na\endcsname{\def\PY@tc##1{\textcolor[rgb]{0.49,0.56,0.16}{##1}}}
\expandafter\def\csname PY@tok@nt\endcsname{\let\PY@bf=\textbf\def\PY@tc##1{\textcolor[rgb]{0.00,0.50,0.00}{##1}}}
\expandafter\def\csname PY@tok@nd\endcsname{\def\PY@tc##1{\textcolor[rgb]{0.67,0.13,1.00}{##1}}}
\expandafter\def\csname PY@tok@s\endcsname{\def\PY@tc##1{\textcolor[rgb]{0.73,0.13,0.13}{##1}}}
\expandafter\def\csname PY@tok@sd\endcsname{\let\PY@it=\textit\def\PY@tc##1{\textcolor[rgb]{0.73,0.13,0.13}{##1}}}
\expandafter\def\csname PY@tok@si\endcsname{\let\PY@bf=\textbf\def\PY@tc##1{\textcolor[rgb]{0.73,0.40,0.53}{##1}}}
\expandafter\def\csname PY@tok@se\endcsname{\let\PY@bf=\textbf\def\PY@tc##1{\textcolor[rgb]{0.73,0.40,0.13}{##1}}}
\expandafter\def\csname PY@tok@sr\endcsname{\def\PY@tc##1{\textcolor[rgb]{0.73,0.40,0.53}{##1}}}
\expandafter\def\csname PY@tok@ss\endcsname{\def\PY@tc##1{\textcolor[rgb]{0.10,0.09,0.49}{##1}}}
\expandafter\def\csname PY@tok@sx\endcsname{\def\PY@tc##1{\textcolor[rgb]{0.00,0.50,0.00}{##1}}}
\expandafter\def\csname PY@tok@m\endcsname{\def\PY@tc##1{\textcolor[rgb]{0.40,0.40,0.40}{##1}}}
\expandafter\def\csname PY@tok@gh\endcsname{\let\PY@bf=\textbf\def\PY@tc##1{\textcolor[rgb]{0.00,0.00,0.50}{##1}}}
\expandafter\def\csname PY@tok@gu\endcsname{\let\PY@bf=\textbf\def\PY@tc##1{\textcolor[rgb]{0.50,0.00,0.50}{##1}}}
\expandafter\def\csname PY@tok@gd\endcsname{\def\PY@tc##1{\textcolor[rgb]{0.63,0.00,0.00}{##1}}}
\expandafter\def\csname PY@tok@gi\endcsname{\def\PY@tc##1{\textcolor[rgb]{0.00,0.63,0.00}{##1}}}
\expandafter\def\csname PY@tok@gr\endcsname{\def\PY@tc##1{\textcolor[rgb]{1.00,0.00,0.00}{##1}}}
\expandafter\def\csname PY@tok@ge\endcsname{\let\PY@it=\textit}
\expandafter\def\csname PY@tok@gs\endcsname{\let\PY@bf=\textbf}
\expandafter\def\csname PY@tok@gp\endcsname{\let\PY@bf=\textbf\def\PY@tc##1{\textcolor[rgb]{0.00,0.00,0.50}{##1}}}
\expandafter\def\csname PY@tok@go\endcsname{\def\PY@tc##1{\textcolor[rgb]{0.53,0.53,0.53}{##1}}}
\expandafter\def\csname PY@tok@gt\endcsname{\def\PY@tc##1{\textcolor[rgb]{0.00,0.27,0.87}{##1}}}
\expandafter\def\csname PY@tok@err\endcsname{\def\PY@bc##1{\setlength{\fboxsep}{0pt}\fcolorbox[rgb]{1.00,0.00,0.00}{1,1,1}{\strut ##1}}}
\expandafter\def\csname PY@tok@kc\endcsname{\let\PY@bf=\textbf\def\PY@tc##1{\textcolor[rgb]{0.00,0.50,0.00}{##1}}}
\expandafter\def\csname PY@tok@kd\endcsname{\let\PY@bf=\textbf\def\PY@tc##1{\textcolor[rgb]{0.00,0.50,0.00}{##1}}}
\expandafter\def\csname PY@tok@kn\endcsname{\let\PY@bf=\textbf\def\PY@tc##1{\textcolor[rgb]{0.00,0.50,0.00}{##1}}}
\expandafter\def\csname PY@tok@kr\endcsname{\let\PY@bf=\textbf\def\PY@tc##1{\textcolor[rgb]{0.00,0.50,0.00}{##1}}}
\expandafter\def\csname PY@tok@bp\endcsname{\def\PY@tc##1{\textcolor[rgb]{0.00,0.50,0.00}{##1}}}
\expandafter\def\csname PY@tok@fm\endcsname{\def\PY@tc##1{\textcolor[rgb]{0.00,0.00,1.00}{##1}}}
\expandafter\def\csname PY@tok@vc\endcsname{\def\PY@tc##1{\textcolor[rgb]{0.10,0.09,0.49}{##1}}}
\expandafter\def\csname PY@tok@vg\endcsname{\def\PY@tc##1{\textcolor[rgb]{0.10,0.09,0.49}{##1}}}
\expandafter\def\csname PY@tok@vi\endcsname{\def\PY@tc##1{\textcolor[rgb]{0.10,0.09,0.49}{##1}}}
\expandafter\def\csname PY@tok@vm\endcsname{\def\PY@tc##1{\textcolor[rgb]{0.10,0.09,0.49}{##1}}}
\expandafter\def\csname PY@tok@sa\endcsname{\def\PY@tc##1{\textcolor[rgb]{0.73,0.13,0.13}{##1}}}
\expandafter\def\csname PY@tok@sb\endcsname{\def\PY@tc##1{\textcolor[rgb]{0.73,0.13,0.13}{##1}}}
\expandafter\def\csname PY@tok@sc\endcsname{\def\PY@tc##1{\textcolor[rgb]{0.73,0.13,0.13}{##1}}}
\expandafter\def\csname PY@tok@dl\endcsname{\def\PY@tc##1{\textcolor[rgb]{0.73,0.13,0.13}{##1}}}
\expandafter\def\csname PY@tok@s2\endcsname{\def\PY@tc##1{\textcolor[rgb]{0.73,0.13,0.13}{##1}}}
\expandafter\def\csname PY@tok@sh\endcsname{\def\PY@tc##1{\textcolor[rgb]{0.73,0.13,0.13}{##1}}}
\expandafter\def\csname PY@tok@s1\endcsname{\def\PY@tc##1{\textcolor[rgb]{0.73,0.13,0.13}{##1}}}
\expandafter\def\csname PY@tok@mb\endcsname{\def\PY@tc##1{\textcolor[rgb]{0.40,0.40,0.40}{##1}}}
\expandafter\def\csname PY@tok@mf\endcsname{\def\PY@tc##1{\textcolor[rgb]{0.40,0.40,0.40}{##1}}}
\expandafter\def\csname PY@tok@mh\endcsname{\def\PY@tc##1{\textcolor[rgb]{0.40,0.40,0.40}{##1}}}
\expandafter\def\csname PY@tok@mi\endcsname{\def\PY@tc##1{\textcolor[rgb]{0.40,0.40,0.40}{##1}}}
\expandafter\def\csname PY@tok@il\endcsname{\def\PY@tc##1{\textcolor[rgb]{0.40,0.40,0.40}{##1}}}
\expandafter\def\csname PY@tok@mo\endcsname{\def\PY@tc##1{\textcolor[rgb]{0.40,0.40,0.40}{##1}}}
\expandafter\def\csname PY@tok@ch\endcsname{\let\PY@it=\textit\def\PY@tc##1{\textcolor[rgb]{0.25,0.50,0.50}{##1}}}
\expandafter\def\csname PY@tok@cm\endcsname{\let\PY@it=\textit\def\PY@tc##1{\textcolor[rgb]{0.25,0.50,0.50}{##1}}}
\expandafter\def\csname PY@tok@cpf\endcsname{\let\PY@it=\textit\def\PY@tc##1{\textcolor[rgb]{0.25,0.50,0.50}{##1}}}
\expandafter\def\csname PY@tok@c1\endcsname{\let\PY@it=\textit\def\PY@tc##1{\textcolor[rgb]{0.25,0.50,0.50}{##1}}}
\expandafter\def\csname PY@tok@cs\endcsname{\let\PY@it=\textit\def\PY@tc##1{\textcolor[rgb]{0.25,0.50,0.50}{##1}}}

\def\PYZbs{\char`\\}
\def\PYZus{\char`\_}
\def\PYZob{\char`\{}
\def\PYZcb{\char`\}}
\def\PYZca{\char`\^}
\def\PYZam{\char`\&}
\def\PYZlt{\char`\<}
\def\PYZgt{\char`\>}
\def\PYZsh{\char`\#}
\def\PYZpc{\char`\%}
\def\PYZdl{\char`\$}
\def\PYZhy{\char`\-}
\def\PYZsq{\char`\'}
\def\PYZdq{\char`\"}
\def\PYZti{\char`\~}
% for compatibility with earlier versions
\def\PYZat{@}
\def\PYZlb{[}
\def\PYZrb{]}
\makeatother


    % Exact colors from NB
    \definecolor{incolor}{rgb}{0.0, 0.0, 0.5}
    \definecolor{outcolor}{rgb}{0.545, 0.0, 0.0}



    
    % Prevent overflowing lines due to hard-to-break entities
    \sloppy 
    % Setup hyperref package
    \hypersetup{
      breaklinks=true,  % so long urls are correctly broken across lines
      colorlinks=true,
      urlcolor=urlcolor,
      linkcolor=linkcolor,
      citecolor=citecolor,
      }
    % Slightly bigger margins than the latex defaults
    
    \geometry{verbose,tmargin=1in,bmargin=1in,lmargin=1in,rmargin=1in}
    
    

    \begin{document}
    
    
    \maketitle
    
    

    
    \begin{Verbatim}[commandchars=\\\{\}]
{\color{incolor}In [{\color{incolor}50}]:} \PY{k+kn}{import} \PY{n+nn}{numpy} \PY{k}{as} \PY{n+nn}{np}
         \PY{k+kn}{import} \PY{n+nn}{pandas} \PY{k}{as} \PY{n+nn}{pd}
         \PY{k+kn}{from} \PY{n+nn}{sklearn} \PY{k}{import} \PY{n}{preprocessing}
         \PY{k+kn}{import} \PY{n+nn}{matplotlib}\PY{n+nn}{.}\PY{n+nn}{pyplot} \PY{k}{as} \PY{n+nn}{plt}
         \PY{k+kn}{import} \PY{n+nn}{matplotlib}
         \PY{o}{\PYZpc{}}\PY{k}{matplotlib} inline
         
         \PY{c+c1}{\PYZsh{}}
         \PY{c+c1}{\PYZsh{} TODO: Parameters to play around with}
         \PY{n}{PLOT\PYZus{}TYPE\PYZus{}TEXT} \PY{o}{=} \PY{k+kc}{False} \PY{c+c1}{\PYZsh{} if you\PYZsq{}d like to see indices}
         \PY{n}{PLOT\PYZus{}VECTORS} \PY{o}{=} \PY{k+kc}{True}    \PY{c+c1}{\PYZsh{} if you\PYZsq{}d like to see your original features in P.C.\PYZhy{}Space}
         
         \PY{n}{matplotlib}\PY{o}{.}\PY{n}{style}\PY{o}{.}\PY{n}{use}\PY{p}{(}\PY{l+s+s1}{\PYZsq{}}\PY{l+s+s1}{ggplot}\PY{l+s+s1}{\PYZsq{}}\PY{p}{)} \PY{c+c1}{\PYZsh{} Look Pretty}
         \PY{n}{c} \PY{o}{=} \PY{p}{[}\PY{l+s+s1}{\PYZsq{}}\PY{l+s+s1}{red}\PY{l+s+s1}{\PYZsq{}}\PY{p}{,} \PY{l+s+s1}{\PYZsq{}}\PY{l+s+s1}{green}\PY{l+s+s1}{\PYZsq{}}\PY{p}{,} \PY{l+s+s1}{\PYZsq{}}\PY{l+s+s1}{blue}\PY{l+s+s1}{\PYZsq{}}\PY{p}{,} \PY{l+s+s1}{\PYZsq{}}\PY{l+s+s1}{orange}\PY{l+s+s1}{\PYZsq{}}\PY{p}{,} \PY{l+s+s1}{\PYZsq{}}\PY{l+s+s1}{yellow}\PY{l+s+s1}{\PYZsq{}}\PY{p}{,} \PY{l+s+s1}{\PYZsq{}}\PY{l+s+s1}{brown}\PY{l+s+s1}{\PYZsq{}}\PY{p}{]}
         
         \PY{k}{def} \PY{n+nf}{drawVectors}\PY{p}{(}\PY{n}{transformed\PYZus{}features}\PY{p}{,} \PY{n}{components\PYZus{}}\PY{p}{,} \PY{n}{columns}\PY{p}{,} \PY{n}{plt}\PY{p}{)}\PY{p}{:}
             \PY{n}{num\PYZus{}columns} \PY{o}{=} \PY{n+nb}{len}\PY{p}{(}\PY{n}{columns}\PY{p}{)}
             
             \PY{c+c1}{\PYZsh{} This function will project your *original* feature (columns)}
             \PY{c+c1}{\PYZsh{} onto your principal component feature\PYZhy{}space, so that you can}
             \PY{c+c1}{\PYZsh{} visualize how \PYZdq{}important\PYZdq{} each one was in the}
             \PY{c+c1}{\PYZsh{} multi\PYZhy{}dimentional scaling}
             
             \PY{c+c1}{\PYZsh{} Scale the princial components by the max value in the}
             \PY{c+c1}{\PYZsh{} transformed set belogging to that component.}
             \PY{n}{xvector} \PY{o}{=} \PY{n}{components\PYZus{}}\PY{p}{[}\PY{l+m+mi}{0}\PY{p}{]} \PY{o}{*} \PY{n+nb}{max}\PY{p}{(}\PY{n}{transformed\PYZus{}features}\PY{p}{[}\PY{p}{:}\PY{p}{,}\PY{l+m+mi}{0}\PY{p}{]}\PY{p}{)}
             \PY{n}{yvector} \PY{o}{=} \PY{n}{components\PYZus{}}\PY{p}{[}\PY{l+m+mi}{1}\PY{p}{]} \PY{o}{*} \PY{n+nb}{max}\PY{p}{(}\PY{n}{transformed\PYZus{}features}\PY{p}{[}\PY{p}{:}\PY{p}{,}\PY{l+m+mi}{1}\PY{p}{]}\PY{p}{)}
             
             \PY{c+c1}{\PYZsh{} Visualize projections}
             
             \PY{c+c1}{\PYZsh{} Sort each column by its length. These are your *original*}
             \PY{c+c1}{\PYZsh{} columns, not the principal components.}
             \PY{k+kn}{import} \PY{n+nn}{math}
             \PY{n}{important\PYZus{}features} \PY{o}{=} \PY{p}{\PYZob{}} \PY{n}{columns}\PY{p}{[}\PY{n}{i}\PY{p}{]} \PY{p}{:} \PY{n}{math}\PY{o}{.}\PY{n}{sqrt}\PY{p}{(}\PY{n}{xvector}\PY{p}{[}\PY{n}{i}\PY{p}{]}\PY{o}{*}\PY{o}{*}\PY{l+m+mi}{2} \PY{o}{+} \PY{n}{yvector}\PY{p}{[}\PY{n}{i}\PY{p}{]}\PY{o}{*}\PY{o}{*}\PY{l+m+mi}{2}\PY{p}{)} \PY{k}{for} \PY{n}{i} \PY{o+ow}{in} \PY{n+nb}{range}\PY{p}{(}\PY{n}{num\PYZus{}columns}\PY{p}{)} \PY{p}{\PYZcb{}}
             \PY{n}{important\PYZus{}features} \PY{o}{=} \PY{n+nb}{sorted}\PY{p}{(}\PY{n+nb}{zip}\PY{p}{(}\PY{n}{important\PYZus{}features}\PY{o}{.}\PY{n}{values}\PY{p}{(}\PY{p}{)}\PY{p}{,} \PY{n}{important\PYZus{}features}\PY{o}{.}\PY{n}{keys}\PY{p}{(}\PY{p}{)}\PY{p}{)}\PY{p}{,} \PY{n}{reverse}\PY{o}{=}\PY{k+kc}{True}\PY{p}{)}
             \PY{n+nb}{print} \PY{p}{(}\PY{l+s+s2}{\PYZdq{}}\PY{l+s+s2}{Projected Features by importance:}\PY{l+s+se}{\PYZbs{}n}\PY{l+s+s2}{\PYZdq{}}\PY{p}{,} \PY{n}{important\PYZus{}features}\PY{p}{)}
             
             \PY{n}{ax} \PY{o}{=} \PY{n}{plt}\PY{o}{.}\PY{n}{axes}\PY{p}{(}\PY{p}{)}
             
             \PY{k}{for} \PY{n}{i} \PY{o+ow}{in} \PY{n+nb}{range}\PY{p}{(}\PY{n}{num\PYZus{}columns}\PY{p}{)}\PY{p}{:}
                 \PY{c+c1}{\PYZsh{} Use an arrow to projet each original feature as a}
                 \PY{c+c1}{\PYZsh{} labeled vector on your principal component axes}
                 \PY{n}{plt}\PY{o}{.}\PY{n}{arrow}\PY{p}{(}\PY{l+m+mi}{0}\PY{p}{,} \PY{l+m+mi}{0}\PY{p}{,} \PY{n}{xvector}\PY{p}{[}\PY{n}{i}\PY{p}{]}\PY{p}{,} \PY{n}{yvector}\PY{p}{[}\PY{n}{i}\PY{p}{]}\PY{p}{,} \PY{n}{color}\PY{o}{=}\PY{l+s+s1}{\PYZsq{}}\PY{l+s+s1}{b}\PY{l+s+s1}{\PYZsq{}}\PY{p}{,} \PY{n}{width}\PY{o}{=}\PY{l+m+mf}{0.0005}\PY{p}{,} \PY{n}{head\PYZus{}width}\PY{o}{=}\PY{l+m+mf}{0.02}\PY{p}{,} \PY{n}{alpha}\PY{o}{=}\PY{l+m+mf}{0.7}\PY{p}{,} \PY{n}{zorder}\PY{o}{=}\PY{l+m+mi}{600000}\PY{p}{)}
                 \PY{n}{plt}\PY{o}{.}\PY{n}{text}\PY{p}{(}\PY{n}{xvector}\PY{p}{[}\PY{n}{i}\PY{p}{]}\PY{o}{*}\PY{l+m+mf}{1.2}\PY{p}{,} \PY{n}{yvector}\PY{p}{[}\PY{n}{i}\PY{p}{]}\PY{o}{*}\PY{l+m+mf}{1.2}\PY{p}{,} \PY{n+nb}{list}\PY{p}{(}\PY{n}{columns}\PY{p}{)}\PY{p}{[}\PY{n}{i}\PY{p}{]}\PY{p}{,} \PY{n}{color}\PY{o}{=}\PY{l+s+s1}{\PYZsq{}}\PY{l+s+s1}{b}\PY{l+s+s1}{\PYZsq{}}\PY{p}{,} \PY{n}{alpha}\PY{o}{=}\PY{l+m+mf}{0.75}\PY{p}{,} \PY{n}{zorder}\PY{o}{=}\PY{l+m+mi}{600000}\PY{p}{)}
                 
             \PY{k}{return} \PY{n}{ax}
         
         \PY{k}{def} \PY{n+nf}{doPCA}\PY{p}{(}\PY{n}{data}\PY{p}{,} \PY{n}{dimensions}\PY{o}{=}\PY{l+m+mi}{2}\PY{p}{)}\PY{p}{:}
             \PY{k+kn}{from} \PY{n+nn}{sklearn}\PY{n+nn}{.}\PY{n+nn}{decomposition} \PY{k}{import} \PY{n}{RandomizedPCA}
             \PY{n}{model} \PY{o}{=} \PY{n}{RandomizedPCA}\PY{p}{(}\PY{n}{n\PYZus{}components}\PY{o}{=}\PY{n}{dimensions}\PY{p}{)}
             \PY{n}{model}\PY{o}{.}\PY{n}{fit}\PY{p}{(}\PY{n}{data}\PY{p}{)}
             \PY{k}{return} \PY{n}{model}
         
         \PY{k}{def} \PY{n+nf}{doKMeans}\PY{p}{(}\PY{n}{data}\PY{p}{,} \PY{n}{clusters}\PY{o}{=}\PY{l+m+mi}{0}\PY{p}{)}\PY{p}{:}
             \PY{c+c1}{\PYZsh{}}
             \PY{c+c1}{\PYZsh{} TODO: do the KMeans clustering here, passing in the \PYZsh{} of clusters parameter}
             \PY{c+c1}{\PYZsh{} and fit it against your data. Then, return a tuple containing the cluster}
             \PY{c+c1}{\PYZsh{} centers and the label}
             \PY{c+c1}{\PYZsh{}}
             \PY{k+kn}{from} \PY{n+nn}{sklearn}\PY{n+nn}{.}\PY{n+nn}{cluster} \PY{k}{import} \PY{n}{KMeans}
             \PY{n}{model} \PY{o}{=} \PY{n}{KMeans}\PY{p}{(}\PY{n}{n\PYZus{}clusters}\PY{o}{=}\PY{n}{clusters}\PY{p}{)}
             \PY{n}{model}\PY{o}{.}\PY{n}{fit}\PY{p}{(}\PY{n}{data}\PY{p}{)}
             \PY{k}{return} \PY{n}{model}\PY{o}{.}\PY{n}{cluster\PYZus{}centers\PYZus{}}\PY{p}{,} \PY{n}{model}\PY{o}{.}\PY{n}{labels\PYZus{}}
\end{Verbatim}


    \begin{Verbatim}[commandchars=\\\{\}]
{\color{incolor}In [{\color{incolor}19}]:} \PY{c+c1}{\PYZsh{}}
         \PY{c+c1}{\PYZsh{} TODO: Load up the datase. It has may or may not have nans in it. Make}
         \PY{c+c1}{\PYZsh{} sure you catch them and destroy them, by setting them to \PYZsq{}0\PYZsq{}: This is valid}
         \PY{c+c1}{\PYZsh{} for this dataset, since if the value is missing, you can assume no \PYZdl{} was spent}
         \PY{c+c1}{\PYZsh{} on it.}
         \PY{c+c1}{\PYZsh{}}
         \PY{n}{df} \PY{o}{=} \PY{n}{pd}\PY{o}{.}\PY{n}{read\PYZus{}csv}\PY{p}{(}\PY{l+s+s1}{\PYZsq{}}\PY{l+s+s1}{Datasets/Wholesale customers data.csv}\PY{l+s+s1}{\PYZsq{}}\PY{p}{)}
         \PY{n}{df}\PY{o}{.}\PY{n}{fillna}\PY{p}{(}\PY{l+m+mi}{0}\PY{p}{)}
         \PY{n}{df}\PY{o}{.}\PY{n}{info}\PY{p}{(}\PY{p}{)}
\end{Verbatim}


    \begin{Verbatim}[commandchars=\\\{\}]
<class 'pandas.core.frame.DataFrame'>
RangeIndex: 440 entries, 0 to 439
Data columns (total 8 columns):
Channel             440 non-null int64
Region              440 non-null int64
Fresh               440 non-null int64
Milk                440 non-null int64
Grocery             440 non-null int64
Frozen              440 non-null int64
Detergents\_Paper    440 non-null int64
Delicassen          440 non-null int64
dtypes: int64(8)
memory usage: 27.6 KB

    \end{Verbatim}

    \begin{Verbatim}[commandchars=\\\{\}]
{\color{incolor}In [{\color{incolor}23}]:} \PY{c+c1}{\PYZsh{} }
         \PY{c+c1}{\PYZsh{} TODO: As instructed, get rid of the \PYZsq{}Chanel\PYZsq{} and \PYZsq{}Region\PYZsq{} columns, since}
         \PY{c+c1}{\PYZsh{} you\PYZsq{}ll be invertigating as if this were a single location wholesaler, rather}
         \PY{c+c1}{\PYZsh{} than a national / international one. Leaving these fields in here would cause}
         \PY{c+c1}{\PYZsh{} KMeans to example and give weight to them.}
         \PY{c+c1}{\PYZsh{}}
         \PY{n}{df} \PY{o}{=} \PY{n}{df}\PY{o}{.}\PY{n}{drop}\PY{p}{(}\PY{p}{[}\PY{l+s+s1}{\PYZsq{}}\PY{l+s+s1}{Channel}\PY{l+s+s1}{\PYZsq{}}\PY{p}{,} \PY{l+s+s1}{\PYZsq{}}\PY{l+s+s1}{Region}\PY{l+s+s1}{\PYZsq{}}\PY{p}{]}\PY{p}{,} \PY{n}{axis}\PY{o}{=}\PY{l+m+mi}{1}\PY{p}{)}
\end{Verbatim}


    \begin{Verbatim}[commandchars=\\\{\}]
{\color{incolor}In [{\color{incolor}27}]:} \PY{c+c1}{\PYZsh{}}
         \PY{c+c1}{\PYZsh{} TODO: Before unitizing / standardizing / normalizing your data in preparation for}
         \PY{c+c1}{\PYZsh{} K\PYZhy{}Means, it\PYZsq{}s a good idead to get a quick peek at it. You can do this using the }
         \PY{c+c1}{\PYZsh{} .describe() method, or even by using the built\PYZhy{}in pandas df.plot.hist()}
         \PY{c+c1}{\PYZsh{}}
         \PY{n+nb}{print}\PY{p}{(}\PY{n}{df}\PY{o}{.}\PY{n}{describe}\PY{p}{(}\PY{p}{)}\PY{p}{)}
         \PY{n}{df}\PY{o}{.}\PY{n}{plot}\PY{o}{.}\PY{n}{hist}\PY{p}{(}\PY{p}{)}
\end{Verbatim}


    \begin{Verbatim}[commandchars=\\\{\}]
               Fresh          Milk       Grocery        Frozen  \textbackslash{}
count     440.000000    440.000000    440.000000    440.000000   
mean    12000.297727   5796.265909   7951.277273   3071.931818   
std     12647.328865   7380.377175   9503.162829   4854.673333   
min         3.000000     55.000000      3.000000     25.000000   
25\%      3127.750000   1533.000000   2153.000000    742.250000   
50\%      8504.000000   3627.000000   4755.500000   1526.000000   
75\%     16933.750000   7190.250000  10655.750000   3554.250000   
max    112151.000000  73498.000000  92780.000000  60869.000000   

       Detergents\_Paper    Delicassen  
count        440.000000    440.000000  
mean        2881.493182   1524.870455  
std         4767.854448   2820.105937  
min            3.000000      3.000000  
25\%          256.750000    408.250000  
50\%          816.500000    965.500000  
75\%         3922.000000   1820.250000  
max        40827.000000  47943.000000  

    \end{Verbatim}

\begin{Verbatim}[commandchars=\\\{\}]
{\color{outcolor}Out[{\color{outcolor}27}]:} <matplotlib.axes.\_subplots.AxesSubplot at 0x1fe2f20c588>
\end{Verbatim}
            
    \begin{center}
    \adjustimage{max size={0.9\linewidth}{0.9\paperheight}}{output_3_2.png}
    \end{center}
    { \hspace*{\fill} \\}
    
    \begin{Verbatim}[commandchars=\\\{\}]
{\color{incolor}In [{\color{incolor}31}]:} \PY{c+c1}{\PYZsh{}}
         \PY{c+c1}{\PYZsh{} TODO: Having checked out your data, you may have noticed there\PYZsq{}s a pretty big gap}
         \PY{c+c1}{\PYZsh{} between the top customers in each feature category and thee rest. Some feature}
         \PY{c+c1}{\PYZsh{} scalling algo won\PYZsq{}t get rid of outliers for you, so it\PYZsq{}s a good idea to handle that}
         \PY{c+c1}{\PYZsh{} manually\PYZhy{}\PYZhy{}\PYZhy{}particularly if your data is NOT to determine the top customers. After}
         \PY{c+c1}{\PYZsh{} all, you can do that with a simple Pandas .sort\PYZus{}values() and not a machine learning}
         \PY{c+c1}{\PYZsh{} clustering algorithm. From a business perspective, you\PYZsq{}re probably more interested}
         \PY{c+c1}{\PYZsh{} in clustering your +:\PYZhy{} 2 standards deviation customers, rather than the creme delta}
         \PY{c+c1}{\PYZsh{} creme, or botton of the barrel\PYZsq{}ers}
         \PY{c+c1}{\PYZsh{}}
         \PY{c+c1}{\PYZsh{} Remove top 5 and bottom 5 samples for each column:}
         \PY{n}{drop} \PY{o}{=} \PY{p}{\PYZob{}}\PY{p}{\PYZcb{}}
         \PY{k}{for} \PY{n}{col} \PY{o+ow}{in} \PY{n}{df}\PY{o}{.}\PY{n}{columns}\PY{p}{:}
             \PY{c+c1}{\PYZsh{} Bottom 5}
             \PY{n}{sort} \PY{o}{=} \PY{n}{df}\PY{o}{.}\PY{n}{sort\PYZus{}values}\PY{p}{(}\PY{n}{by}\PY{o}{=}\PY{n}{col}\PY{p}{,} \PY{n}{ascending}\PY{o}{=}\PY{k+kc}{True}\PY{p}{)}
             \PY{k}{if} \PY{n+nb}{len}\PY{p}{(}\PY{n}{sort}\PY{p}{)} \PY{o}{\PYZgt{}} \PY{l+m+mi}{5}\PY{p}{:} \PY{n}{sort}\PY{o}{=}\PY{n}{sort}\PY{p}{[}\PY{p}{:}\PY{l+m+mi}{5}\PY{p}{]}
             \PY{k}{for} \PY{n}{index} \PY{o+ow}{in} \PY{n}{sort}\PY{o}{.}\PY{n}{index}\PY{p}{:} \PY{n}{drop}\PY{p}{[}\PY{n}{index}\PY{p}{]} \PY{o}{=} \PY{k+kc}{True} \PY{c+c1}{\PYZsh{} Just store the index one}
             
             \PY{c+c1}{\PYZsh{} Top 5:}
             \PY{n}{sort} \PY{o}{=} \PY{n}{df}\PY{o}{.}\PY{n}{sort\PYZus{}values}\PY{p}{(}\PY{n}{by}\PY{o}{=}\PY{n}{col}\PY{p}{,} \PY{n}{ascending}\PY{o}{=}\PY{k+kc}{False}\PY{p}{)}
             \PY{k}{if} \PY{n+nb}{len}\PY{p}{(}\PY{n}{sort}\PY{p}{)} \PY{o}{\PYZgt{}} \PY{l+m+mi}{5}\PY{p}{:} \PY{n}{sort}\PY{o}{=}\PY{n}{sort}\PY{p}{[}\PY{p}{:}\PY{l+m+mi}{5}\PY{p}{]}
             \PY{k}{for} \PY{n}{index} \PY{o+ow}{in} \PY{n}{sort}\PY{o}{.}\PY{n}{index}\PY{p}{:} \PY{n}{drop}\PY{p}{[}\PY{n}{index}\PY{p}{]} \PY{o}{=} \PY{k+kc}{True} \PY{c+c1}{\PYZsh{} Just store the index one}
\end{Verbatim}


    \begin{Verbatim}[commandchars=\\\{\}]
{\color{incolor}In [{\color{incolor}33}]:} \PY{c+c1}{\PYZsh{}}
         \PY{c+c1}{\PYZsh{} INFO Drop rows by index. We do this all at once in case there is a}
         \PY{c+c1}{\PYZsh{} collision. This way, we don\PYZsq{}t end up dropping more rows than we have}
         \PY{c+c1}{\PYZsh{} to, if there is a single row that satisfies the drop for multiple columns.}
         \PY{c+c1}{\PYZsh{} Since there are 6 rows, if we end up dropping \PYZlt{} 5*6*2 = 60 rows, that means}
         \PY{c+c1}{\PYZsh{} there indeed were collisions.}
         \PY{n+nb}{print} \PY{p}{(}\PY{l+s+s2}{\PYZdq{}}\PY{l+s+s2}{Dropping }\PY{l+s+si}{\PYZob{}0\PYZcb{}}\PY{l+s+s2}{ Outliers...}\PY{l+s+s2}{\PYZdq{}}\PY{o}{.}\PY{n}{format}\PY{p}{(}\PY{n+nb}{len}\PY{p}{(}\PY{n}{drop}\PY{p}{)}\PY{p}{)}\PY{p}{)}
         \PY{n}{df}\PY{o}{.}\PY{n}{drop}\PY{p}{(}\PY{n}{inplace}\PY{o}{=}\PY{k+kc}{True}\PY{p}{,} \PY{n}{labels}\PY{o}{=}\PY{n}{drop}\PY{o}{.}\PY{n}{keys}\PY{p}{(}\PY{p}{)}\PY{p}{,} \PY{n}{axis}\PY{o}{=}\PY{l+m+mi}{0}\PY{p}{)}
         \PY{n+nb}{print} \PY{p}{(}\PY{n}{df}\PY{o}{.}\PY{n}{describe}\PY{p}{(}\PY{p}{)}\PY{p}{)}
\end{Verbatim}


    \begin{Verbatim}[commandchars=\\\{\}]
Dropping 42 Outliers{\ldots}
              Fresh          Milk       Grocery        Frozen  \textbackslash{}
count    398.000000    398.000000    398.000000    398.000000   
mean   10996.231156   5144.090452   7091.711055   2639.721106   
std     9933.042596   5057.406574   6923.019293   2974.246906   
min       37.000000    258.000000    314.000000     47.000000   
25\%     3324.500000   1571.250000   2155.500000    749.750000   
50\%     8257.500000   3607.500000   4573.000000   1526.000000   
75\%    15828.500000   6953.250000   9922.250000   3370.250000   
max    53205.000000  29892.000000  39694.000000  17866.000000   

       Detergents\_Paper   Delicassen  
count        398.000000   398.000000  
mean        2562.974874  1278.736181  
std         3608.176776  1220.745297  
min           10.000000    11.000000  
25\%          273.250000   409.500000  
50\%          812.000000   946.500000  
75\%         3841.500000  1752.250000  
max        19410.000000  7844.000000  

    \end{Verbatim}

    \begin{Verbatim}[commandchars=\\\{\}]
{\color{incolor}In [{\color{incolor}34}]:} \PY{c+c1}{\PYZsh{}}
         \PY{c+c1}{\PYZsh{} INFO: What are you interested in?}
         \PY{c+c1}{\PYZsh{}}
         \PY{c+c1}{\PYZsh{} Depending on what you\PYZsq{}re interested in, you might take a different approach}
         \PY{c+c1}{\PYZsh{} to normalizing/standardizing your data.}
         \PY{c+c1}{\PYZsh{} }
         \PY{c+c1}{\PYZsh{} You should note that all columns left in the dataset are of the same unit.}
         \PY{c+c1}{\PYZsh{} You might ask yourself, do I even need to normalize / standardize the data?}
         \PY{c+c1}{\PYZsh{} The answer depends on what you\PYZsq{}re trying to accomplish. For instance, although}
         \PY{c+c1}{\PYZsh{} all the units are the same (generic money unit), the price per item in your}
         \PY{c+c1}{\PYZsh{} store isn\PYZsq{}t. There may be some cheap items and some expensive one. If your goal}
         \PY{c+c1}{\PYZsh{} is to find out what items people buy tend to buy together but you didn\PYZsq{}t }
         \PY{c+c1}{\PYZsh{} unitize properly before running kMeans, the contribution of the lesser priced}
         \PY{c+c1}{\PYZsh{} item would be dwarfed by the more expensive item.}
         \PY{c+c1}{\PYZsh{}}
         \PY{c+c1}{\PYZsh{} For a great overview on a few of the normalization methods supported in SKLearn,}
         \PY{c+c1}{\PYZsh{} please check out: https://stackoverflow.com/questions/30918781/right\PYZhy{}function\PYZhy{}for\PYZhy{}normalizing\PYZhy{}input\PYZhy{}of\PYZhy{}sklearn\PYZhy{}svm}
         \PY{c+c1}{\PYZsh{}}
         \PY{c+c1}{\PYZsh{} Suffice to say, at the end of the day, you\PYZsq{}re going to have to know what question}
         \PY{c+c1}{\PYZsh{} you want answered and what data you have available in order to select the best}
         \PY{c+c1}{\PYZsh{} method for your purpose. Luckily, SKLearn\PYZsq{}s interfaces are easy to switch out}
         \PY{c+c1}{\PYZsh{} so in the mean time, you can experiment with all of them and see how they alter}
         \PY{c+c1}{\PYZsh{} your results.}
         \PY{c+c1}{\PYZsh{}}
         \PY{c+c1}{\PYZsh{}}
         \PY{c+c1}{\PYZsh{} 5\PYZhy{}sec summary before you dive deeper online:}
         \PY{c+c1}{\PYZsh{}}
         \PY{c+c1}{\PYZsh{} NORMALIZATION: Let\PYZsq{}s say your user spend a LOT. Normalization divides each item by}
         \PY{c+c1}{\PYZsh{}                the average overall amount of spending. Stated differently, your}
         \PY{c+c1}{\PYZsh{}                new feature is = the contribution of overall spending going into}
         \PY{c+c1}{\PYZsh{}                that particular item: \PYZdl{}spent on feature / \PYZdl{}overall spent by sample}
         \PY{c+c1}{\PYZsh{}}
         \PY{c+c1}{\PYZsh{} MINMAX:        What \PYZpc{} in the overall range of \PYZdl{}spent by all users on THIS particular}
         \PY{c+c1}{\PYZsh{}                feature is the current sample\PYZsq{}s feature at? When you\PYZsq{}re dealing with}
         \PY{c+c1}{\PYZsh{}                all the same units, this will produce a near face\PYZhy{}value amount. Be}
         \PY{c+c1}{\PYZsh{}                careful though: if you have even a single outlier, it can cause all}
         \PY{c+c1}{\PYZsh{}                your data to get squashed up in lower percentages.}
         \PY{c+c1}{\PYZsh{}                Imagine your buyers usually spend \PYZdl{}100 on wholesale milk, but today}
         \PY{c+c1}{\PYZsh{}                only spent \PYZdl{}20. This is the relationship you\PYZsq{}re trying to capture }
         \PY{c+c1}{\PYZsh{}                with MinMax. NOTE: MinMax doesn\PYZsq{}t standardize (std. dev.); it only}
         \PY{c+c1}{\PYZsh{}                normalizes / unitizes your feature, in the mathematical sense.}
         \PY{c+c1}{\PYZsh{}                MinMax can be used as an alternative to zero mean, unit variance scaling.}
         \PY{c+c1}{\PYZsh{}                [(sampleFeatureValue\PYZhy{}min) / (max\PYZhy{}min)] * (max\PYZhy{}min) + min}
         \PY{c+c1}{\PYZsh{}                Where min and max are for the overall feature values for all samples.}
\end{Verbatim}


    \begin{Verbatim}[commandchars=\\\{\}]
{\color{incolor}In [{\color{incolor}35}]:} \PY{c+c1}{\PYZsh{}}
         \PY{c+c1}{\PYZsh{} TODO: Un\PYZhy{}comment just ***ONE*** of lines at a time and see how alters your results}
         \PY{c+c1}{\PYZsh{} Pay attention to the direction of the arrows, as well as their LENGTHS}
         \PY{n}{T} \PY{o}{=} \PY{n}{preprocessing}\PY{o}{.}\PY{n}{StandardScaler}\PY{p}{(}\PY{p}{)}\PY{o}{.}\PY{n}{fit\PYZus{}transform}\PY{p}{(}\PY{n}{df}\PY{p}{)}
         \PY{c+c1}{\PYZsh{}T = preprocessing.MinMaxScaler().fit\PYZus{}transform(df)}
         \PY{c+c1}{\PYZsh{}T = preprocessing.MaxAbsScaler().fit\PYZus{}transform(df)}
         \PY{c+c1}{\PYZsh{}T = preprocessing.Normalizer().fit\PYZus{}transform(df)}
         \PY{c+c1}{\PYZsh{}T = df \PYZsh{} No Change}
\end{Verbatim}


    \begin{Verbatim}[commandchars=\\\{\}]
{\color{incolor}In [{\color{incolor}40}]:} \PY{c+c1}{\PYZsh{}}
         \PY{c+c1}{\PYZsh{} INFO: Sometimes people perform PCA before doing KMeans, so that KMeans only}
         \PY{c+c1}{\PYZsh{} operates on the most meaningful features. In our case, there are so few features}
         \PY{c+c1}{\PYZsh{} that doing PCA ahead of time isn\PYZsq{}t really necessary, and you can do KMeans in}
         \PY{c+c1}{\PYZsh{} feature space. But keep in mind you have the option to transform your data to}
         \PY{c+c1}{\PYZsh{} bring down its dimensionality. If you take that route, then your Clusters will}
         \PY{c+c1}{\PYZsh{} already be in PCA\PYZhy{}transformed feature space, and you won\PYZsq{}t have to project them}
         \PY{c+c1}{\PYZsh{} again for visualization.}
         
         
         \PY{c+c1}{\PYZsh{} Do KMeans}
         \PY{n}{n\PYZus{}clusters} \PY{o}{=} \PY{l+m+mi}{3}
         \PY{n}{centroids}\PY{p}{,} \PY{n}{labels} \PY{o}{=} \PY{n}{doKMeans}\PY{p}{(}\PY{n}{T}\PY{p}{,} \PY{n}{n\PYZus{}clusters}\PY{p}{)}
\end{Verbatim}


    \begin{Verbatim}[commandchars=\\\{\}]
{\color{incolor}In [{\color{incolor}42}]:} \PY{c+c1}{\PYZsh{}}
         \PY{c+c1}{\PYZsh{} TODO: Print out your centroids. They\PYZsq{}re currently in feature\PYZhy{}space, which}
         \PY{c+c1}{\PYZsh{} is good. Print them out before you transform them into PCA space for viewing}
         \PY{c+c1}{\PYZsh{}}
         \PY{n+nb}{print}\PY{p}{(}\PY{n}{centroids}\PY{p}{)}
\end{Verbatim}


    \begin{Verbatim}[commandchars=\\\{\}]
[[-0.26232459 -0.41632038 -0.43893222 -0.29733807 -0.38963897 -0.32590196]
 [-0.56995352  1.31744707  1.58541223 -0.33454978  1.59481374  0.27849802]
 [ 1.20725644 -0.12927099 -0.31707856  1.08169261 -0.45400669  0.59174102]]

    \end{Verbatim}

    \begin{Verbatim}[commandchars=\\\{\}]
{\color{incolor}In [{\color{incolor}43}]:} \PY{c+c1}{\PYZsh{} Do PCA *after* to visualize the results. Project the centroids as well as }
         \PY{c+c1}{\PYZsh{} the samples into the new 2D feature space for visualization purposes.}
         \PY{n}{display\PYZus{}pca} \PY{o}{=} \PY{n}{doPCA}\PY{p}{(}\PY{n}{T}\PY{p}{)}
         \PY{n}{T} \PY{o}{=} \PY{n}{display\PYZus{}pca}\PY{o}{.}\PY{n}{transform}\PY{p}{(}\PY{n}{T}\PY{p}{)}
         \PY{n}{CC} \PY{o}{=} \PY{n}{display\PYZus{}pca}\PY{o}{.}\PY{n}{transform}\PY{p}{(}\PY{n}{centroids}\PY{p}{)}
\end{Verbatim}


    \begin{Verbatim}[commandchars=\\\{\}]
C:\textbackslash{}Users\textbackslash{}CongCuongPHAM\textbackslash{}AppData\textbackslash{}Local\textbackslash{}Continuum\textbackslash{}anaconda3\textbackslash{}lib\textbackslash{}site-packages\textbackslash{}sklearn\textbackslash{}utils\textbackslash{}deprecation.py:58: DeprecationWarning: Class RandomizedPCA is deprecated; RandomizedPCA was deprecated in 0.18 and will be removed in 0.20. Use PCA(svd\_solver='randomized') instead. The new implementation DOES NOT store whiten ``components\_``. Apply transform to get them.
  warnings.warn(msg, category=DeprecationWarning)

    \end{Verbatim}

    \begin{Verbatim}[commandchars=\\\{\}]
{\color{incolor}In [{\color{incolor}51}]:} \PY{c+c1}{\PYZsh{} Visualize all the samples. Give them the color of their cluster label}
         \PY{n}{fig} \PY{o}{=} \PY{n}{plt}\PY{o}{.}\PY{n}{figure}\PY{p}{(}\PY{p}{)}
         \PY{n}{ax} \PY{o}{=} \PY{n}{fig}\PY{o}{.}\PY{n}{add\PYZus{}subplot}\PY{p}{(}\PY{l+m+mi}{111}\PY{p}{)}
         \PY{k}{if} \PY{n}{PLOT\PYZus{}TYPE\PYZus{}TEXT}\PY{p}{:}
           \PY{c+c1}{\PYZsh{} Plot the index of the sample, so you can further investigate it in your dset}
           \PY{k}{for} \PY{n}{i} \PY{o+ow}{in} \PY{n+nb}{range}\PY{p}{(}\PY{n+nb}{len}\PY{p}{(}\PY{n}{T}\PY{p}{)}\PY{p}{)}\PY{p}{:} \PY{n}{ax}\PY{o}{.}\PY{n}{text}\PY{p}{(}\PY{n}{T}\PY{p}{[}\PY{n}{i}\PY{p}{,}\PY{l+m+mi}{0}\PY{p}{]}\PY{p}{,} \PY{n}{T}\PY{p}{[}\PY{n}{i}\PY{p}{,}\PY{l+m+mi}{1}\PY{p}{]}\PY{p}{,} \PY{n}{df}\PY{o}{.}\PY{n}{index}\PY{p}{[}\PY{n}{i}\PY{p}{]}\PY{p}{,} \PY{n}{color}\PY{o}{=}\PY{n}{c}\PY{p}{[}\PY{n}{labels}\PY{p}{[}\PY{n}{i}\PY{p}{]}\PY{p}{]}\PY{p}{,} \PY{n}{alpha}\PY{o}{=}\PY{l+m+mf}{0.75}\PY{p}{,} \PY{n}{zorder}\PY{o}{=}\PY{l+m+mi}{600000}\PY{p}{)}
           \PY{n}{ax}\PY{o}{.}\PY{n}{set\PYZus{}xlim}\PY{p}{(}\PY{n+nb}{min}\PY{p}{(}\PY{n}{T}\PY{p}{[}\PY{p}{:}\PY{p}{,}\PY{l+m+mi}{0}\PY{p}{]}\PY{p}{)}\PY{o}{*}\PY{l+m+mf}{1.2}\PY{p}{,} \PY{n+nb}{max}\PY{p}{(}\PY{n}{T}\PY{p}{[}\PY{p}{:}\PY{p}{,}\PY{l+m+mi}{0}\PY{p}{]}\PY{p}{)}\PY{o}{*}\PY{l+m+mf}{1.2}\PY{p}{)}
           \PY{n}{ax}\PY{o}{.}\PY{n}{set\PYZus{}ylim}\PY{p}{(}\PY{n+nb}{min}\PY{p}{(}\PY{n}{T}\PY{p}{[}\PY{p}{:}\PY{p}{,}\PY{l+m+mi}{1}\PY{p}{]}\PY{p}{)}\PY{o}{*}\PY{l+m+mf}{1.2}\PY{p}{,} \PY{n+nb}{max}\PY{p}{(}\PY{n}{T}\PY{p}{[}\PY{p}{:}\PY{p}{,}\PY{l+m+mi}{1}\PY{p}{]}\PY{p}{)}\PY{o}{*}\PY{l+m+mf}{1.2}\PY{p}{)}
         \PY{k}{else}\PY{p}{:}
           \PY{c+c1}{\PYZsh{} Plot a regular scatter plot}
           \PY{n}{sample\PYZus{}colors} \PY{o}{=} \PY{p}{[} \PY{n}{c}\PY{p}{[}\PY{n}{labels}\PY{p}{[}\PY{n}{i}\PY{p}{]}\PY{p}{]} \PY{k}{for} \PY{n}{i} \PY{o+ow}{in} \PY{n+nb}{range}\PY{p}{(}\PY{n+nb}{len}\PY{p}{(}\PY{n}{T}\PY{p}{)}\PY{p}{)} \PY{p}{]}
           \PY{n}{ax}\PY{o}{.}\PY{n}{scatter}\PY{p}{(}\PY{n}{T}\PY{p}{[}\PY{p}{:}\PY{p}{,} \PY{l+m+mi}{0}\PY{p}{]}\PY{p}{,} \PY{n}{T}\PY{p}{[}\PY{p}{:}\PY{p}{,} \PY{l+m+mi}{1}\PY{p}{]}\PY{p}{,} \PY{n}{c}\PY{o}{=}\PY{n}{sample\PYZus{}colors}\PY{p}{,} \PY{n}{marker}\PY{o}{=}\PY{l+s+s1}{\PYZsq{}}\PY{l+s+s1}{o}\PY{l+s+s1}{\PYZsq{}}\PY{p}{,} \PY{n}{alpha}\PY{o}{=}\PY{l+m+mf}{0.2}\PY{p}{)}
         
         
         \PY{c+c1}{\PYZsh{} Plot the Centroids as X\PYZsq{}s, and label them}
         \PY{n}{ax}\PY{o}{.}\PY{n}{scatter}\PY{p}{(}\PY{n}{CC}\PY{p}{[}\PY{p}{:}\PY{p}{,} \PY{l+m+mi}{0}\PY{p}{]}\PY{p}{,} \PY{n}{CC}\PY{p}{[}\PY{p}{:}\PY{p}{,} \PY{l+m+mi}{1}\PY{p}{]}\PY{p}{,} \PY{n}{marker}\PY{o}{=}\PY{l+s+s1}{\PYZsq{}}\PY{l+s+s1}{x}\PY{l+s+s1}{\PYZsq{}}\PY{p}{,} \PY{n}{s}\PY{o}{=}\PY{l+m+mi}{169}\PY{p}{,} \PY{n}{linewidths}\PY{o}{=}\PY{l+m+mi}{3}\PY{p}{,} \PY{n}{zorder}\PY{o}{=}\PY{l+m+mi}{1000}\PY{p}{,} \PY{n}{c}\PY{o}{=}\PY{n}{c}\PY{p}{)}
         \PY{k}{for} \PY{n}{i} \PY{o+ow}{in} \PY{n+nb}{range}\PY{p}{(}\PY{n+nb}{len}\PY{p}{(}\PY{n}{centroids}\PY{p}{)}\PY{p}{)}\PY{p}{:} \PY{n}{ax}\PY{o}{.}\PY{n}{text}\PY{p}{(}\PY{n}{CC}\PY{p}{[}\PY{n}{i}\PY{p}{,} \PY{l+m+mi}{0}\PY{p}{]}\PY{p}{,} \PY{n}{CC}\PY{p}{[}\PY{n}{i}\PY{p}{,} \PY{l+m+mi}{1}\PY{p}{]}\PY{p}{,} \PY{n+nb}{str}\PY{p}{(}\PY{n}{i}\PY{p}{)}\PY{p}{,} \PY{n}{zorder}\PY{o}{=}\PY{l+m+mi}{500010}\PY{p}{,} \PY{n}{fontsize}\PY{o}{=}\PY{l+m+mi}{18}\PY{p}{,} \PY{n}{color}\PY{o}{=}\PY{n}{c}\PY{p}{[}\PY{n}{i}\PY{p}{]}\PY{p}{)}
         
         
         \PY{c+c1}{\PYZsh{} Display feature vectors for investigation:}
         \PY{k}{if} \PY{n}{PLOT\PYZus{}VECTORS}\PY{p}{:} \PY{n}{drawVectors}\PY{p}{(}\PY{n}{T}\PY{p}{,} \PY{n}{display\PYZus{}pca}\PY{o}{.}\PY{n}{components\PYZus{}}\PY{p}{,} \PY{n}{df}\PY{o}{.}\PY{n}{columns}\PY{p}{,} \PY{n}{plt}\PY{p}{)}
         
         
         \PY{c+c1}{\PYZsh{} Add the cluster label back into the dataframe and display it:}
         \PY{n}{df}\PY{p}{[}\PY{l+s+s1}{\PYZsq{}}\PY{l+s+s1}{label}\PY{l+s+s1}{\PYZsq{}}\PY{p}{]} \PY{o}{=} \PY{n}{pd}\PY{o}{.}\PY{n}{Series}\PY{p}{(}\PY{n}{labels}\PY{p}{,} \PY{n}{index}\PY{o}{=}\PY{n}{df}\PY{o}{.}\PY{n}{index}\PY{p}{)}
         \PY{n+nb}{print} \PY{p}{(}\PY{n}{df}\PY{p}{)}
         
         \PY{n}{plt}\PY{o}{.}\PY{n}{show}\PY{p}{(}\PY{p}{)}
\end{Verbatim}


    \begin{Verbatim}[commandchars=\\\{\}]
Projected Features by importance:
 [(4.439422524289905, 'Grocery'), (4.3521532757037855, 'Detergents\_Paper'), (4.209841269791215, 'Milk'), (2.94634030532376, 'Delicassen'), (2.6965214173655356, 'Fresh'), (2.6383798432550747, 'Frozen')]
     Fresh   Milk  Grocery  Frozen  Detergents\_Paper  Delicassen  label
0    12669   9656     7561     214              2674        1338      0
1     7057   9810     9568    1762              3293        1776      0
2     6353   8808     7684    2405              3516        7844      2
3    13265   1196     4221    6404               507        1788      2
4    22615   5410     7198    3915              1777        5185      2
5     9413   8259     5126     666              1795        1451      0
6    12126   3199     6975     480              3140         545      0
7     7579   4956     9426    1669              3321        2566      0
8     5963   3648     6192     425              1716         750      0
9     6006  11093    18881    1159              7425        2098      1
10    3366   5403    12974    4400              5977        1744      1
11   13146   1124     4523    1420               549         497      0
12   31714  12319    11757     287              3881        2931      2
13   21217   6208    14982    3095              6707         602      1
14   24653   9465    12091     294              5058        2168      1
15   10253   1114     3821     397               964         412      0
16    1020   8816    12121     134              4508        1080      1
17    5876   6157     2933     839               370        4478      0
18   18601   6327    10099    2205              2767        3181      2
19    7780   2495     9464     669              2518         501      0
20   17546   4519     4602    1066              2259        2124      0
21    5567    871     2010    3383               375         569      0
22   31276   1917     4469    9408              2381        4334      2
24   22647   9776    13792    2915              4482        5778      2
25   16165   4230     7595     201              4003          57      0
26    9898    961     2861    3151               242         833      0
27   14276    803     3045     485               100         518      0
28    4113  20484    25957    1158              8604        5206      1
29   43088   2100     2609    1200              1107         823      2
30   18815   3610    11107    1148              2134        2963      2
..     {\ldots}    {\ldots}      {\ldots}     {\ldots}               {\ldots}         {\ldots}    {\ldots}
409   8708   3634     6100    2349              2123        5137      2
410   6633   2096     4563    1389              1860        1892      0
411   2126   3289     3281    1535               235        4365      0
412     97   3605    12400      98              2970          62      0
413   4983   4859     6633   17866               912        2435      2
414   5969   1990     3417    5679              1135         290      0
415   7842   6046     8552    1691              3540        1874      0
416   4389  10940    10908     848              6728         993      1
417   5065   5499    11055     364              3485        1063      0
418    660   8494    18622     133              6740         776      1
419   8861   3783     2223     633              1580        1521      0
421  17063   4847     9053    1031              3415        1784      0
422  26400   1377     4172     830               948        1218      0
423  17565   3686     4657    1059              1803         668      0
424  16980   2884    12232     874              3213         249      0
425  11243   2408     2593   15348               108        1886      2
426  13134   9347    14316    3141              5079        1894      1
427  31012  16687     5429   15082               439        1163      2
428   3047   5970     4910    2198               850         317      0
429   8607   1750     3580      47                84        2501      0
430   3097   4230    16483     575               241        2080      0
431   8533   5506     5160   13486              1377        1498      2
432  21117   1162     4754     269              1328         395      0
433   1982   3218     1493    1541               356        1449      0
434  16731   3922     7994     688              2371         838      0
435  29703  12051    16027   13135               182        2204      2
436  39228   1431      764    4510                93        2346      2
437  14531  15488    30243     437             14841        1867      1
438  10290   1981     2232    1038               168        2125      0
439   2787   1698     2510      65               477          52      0

[398 rows x 7 columns]

    \end{Verbatim}

    \begin{Verbatim}[commandchars=\\\{\}]
C:\textbackslash{}Users\textbackslash{}CongCuongPHAM\textbackslash{}AppData\textbackslash{}Local\textbackslash{}Continuum\textbackslash{}anaconda3\textbackslash{}lib\textbackslash{}site-packages\textbackslash{}matplotlib\textbackslash{}cbook\textbackslash{}deprecation.py:106: MatplotlibDeprecationWarning: Adding an axes using the same arguments as a previous axes currently reuses the earlier instance.  In a future version, a new instance will always be created and returned.  Meanwhile, this warning can be suppressed, and the future behavior ensured, by passing a unique label to each axes instance.
  warnings.warn(message, mplDeprecation, stacklevel=1)

    \end{Verbatim}

    \begin{center}
    \adjustimage{max size={0.9\linewidth}{0.9\paperheight}}{output_11_2.png}
    \end{center}
    { \hspace*{\fill} \\}
    

    % Add a bibliography block to the postdoc
    
    
    
    \end{document}
